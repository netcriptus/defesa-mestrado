O avanço nas pesquisas de protocolos tem conferido mais confiabilidade e segurança
para redes MANET, permitindo a expansão de seu uso em vários níveis. Muitas das
barreiras e desafios identificados na última década já foram superadas ou apresentaram
uma melhora expressiva \cite{manet-state}.

Os problemas de roteamento foram resolvidos por protocolos relativamente recentes
como \textit{AdHoc On-Demand Distance Vector} (AODV) \cite{aodv} e
\textit{Dynamic Source Routing} (DSR) \cite{dsr}. A Qualidade de Serviço é
frequentemente abordada em vários novos protocolos, \cite{qos1} e \cite{qos2}
demonstram algumas pesquisas nessa área. Segurança e consumo de energia são
preocupações constantes na nova geração de protocolos, fazendo com que a rede
evolua nesses aspectos a cada ano.

\subsection{Problema}

A evolução dos protocolos de tradução de nomes tem sido tímida nessa área. Do
modo como foi projetado em 1987 \cite{rfc1035}, o protocolo original do DNS é
altamente dependente de infraestrutura. Sua especificação prevê uma hierarquia rígida,
sua segurança e confiabilidade parte do pressuposto de servidores estáveis e de
uma rede com pouca volatilidade.

Uma rede MANET elimina os principais elementos usados pelo DNS -- hierarquia em
árvore, IP's fixos associados a nomes, servidores com muitos recursos e pouca
quebra de links -- então é de se esperar que o protocolo em sua forma pura não
seja o suficiente para esse ambiente.

Diferente dos outros desafios descritos em \cite{manet-state}, não existe um
protocolo oficial ou aproximação consensualmente melhor para o problema da
resolução de nomes. Os principais protocolos que se propõe a executar a tarefa
de tradução assumem pelo menos um dos elementos citados anteriormente -- em
especial a presença de um servidor -- o que é prejudicial aos nodos na rede, ou
ao menos prejudicial aos nodos escolhidos para serem servidores.

\subsection{Objetivos}

Para propor um protocolo de tradução de nomes em uma rede que não oferece nada
da estrutura básica requerida pelo protocolo tradicional, é preciso uma nova
abordagem. O novo protocolo precisa assumir que
\begin{inparaenum}[(i)]
    \item não existe servidor fixo permanente;
    \item não existe hierarquia;
    \item um nó, assim como a informação que ele carrega pode sumir a qualquer
    instante;
    \item espaço de armazenamento pode ser limitado (uma vez que podem existir
    equipamentos com poucos recursos na rede);
    \item o número de mensagens trocadas deve ser minimizado e broadcasts devem
    ser evitados, visando economia de energia.
\end{inparaenum}

O protocolo proposto neste trabalho visa atender essas restrições e ser mais
eficiente que seus predecessores \cite{mcdsn} \cite{dnssd} \cite{mdns} nas questões
de consumo de energia e espaço de armazenamento utilizado. A segurança do protocolo
no entanto está fora do escopo deste trabalho.

\subsection{Estrutura da proposta}

Este documento está organizado em quatro capítulos. O capítulo 2 analisa o estado
dos protocolos que existem hoje para resolução de nomes em MANETS, explicando o
funcionamento e apresentando os pontos fortes e fracos de cada protocolo.

O terceiro capítulo descreve a proposta de um novo protocolo completamente
distribuído para tradução de nomes e descoberta de serviços, incluindo os
algoritmos das tarefas executadas e as possibilidades de traduções. Posteriormente
algumas regras sobre conflitos de nomes e desconexão da rede são apresentadas.

O último capítulo apresenta a estratégia proposta para testes. Esse capítulo
explica o funcionamento da plataforma que será usada e por que foi escolhido um
emulador de rede ao invés de um simulador.