%!TEX root = ../main.tex

Este capítulo descreve a metodologia a ser aplicada para avaliação do protocolo
FDNS-SD. A seção \ref{ambiente} descreve o ambiente que será usado nos testes.
A sessão \ref{futuro} apresenta o cronograma com as tarefas e os passos futuros
da proposta. A sessão \ref{conclusao} apresenta as considerações finais.

\section{Ambiente de desenvolvimento}
\label{ambiente}

    Apesar do software de testes mais utilizado para simulações de redes ser o
    \textit{Network Simulator} (NS) versão 2, a opção para testar esse protocolo
    foi outra. Para testar o \textit{FDNS-SD} em condições mais próximas às reais,
    será usado o \textit{Testbed} Orbit Lab \cite{orbit}.
    
    Operado e desenvolvido pelo laboratório \textit{WINLAB} \cite{winlab} da
    \textit{Rutgers University} em conjunto com \textit{Princeton} e
    \textit{Columbia University}, com apoio financeiro da \textit{National
    Science Foundation} (NSF), o Orbit Lab é um emulador de redes móveis, que
    conta com um grid bidimensional de 400 nós \textit{wireless} (802.11).

    A rede do Orbit Lab está preparada para simular inclusive a movimentação dos nós,
    e pode emular diversas topologias. A vantagem de testar o protocolo nesse grid é
    que os dados coletados em questões como overhead e perda de pacotes são reais, e
    não simulados, e assim é possível verificar com mais confiabilidade o comportamento
    do protocolo. O \textit{testbed} Orbit aceita apenas scripts escritos na linguagem
    Ruby.

\section{Cronograma e passos futuros}
\label{futuro}

    O cronograma com as tarefas a serem executadas até a defesa da dissertação
    é apresentado na tabela \ref{cronograma}. Cada passo é explicado em detalhes
    a seguir.
    
    \begin{table}[t]\centering
        \begin{tabular}{ | p{9cm} | c | c | c | c |}
            \hline
            \raggedleft{Atividade} & \begin{sideways}Maio\end{sideways}
             & \begin{sideways}Junho\end{sideways}
             & \begin{sideways}Julho\end{sideways}
             & \begin{sideways}Agosto\end{sideways} \\ \hline
            \raggedleft{Definição dos cenários e métricas para avaliação} & * & * &   &   \\ \hline
            \raggedleft{Implementação do protocolo no Orbit Lab} &   & * & * &   \\ \hline
            \raggedleft{Análise dos resultados} &   &   & * &   \\ \hline
            \raggedleft{Escrita da dissertação} &   & * & * & * \\ \hline
            \raggedleft{Apresentação da dissertação} &   &   &   & * \\ \hline
        \end{tabular}
        \caption{Cronograma proposto de atividades}
        \label{cronograma}
    \end{table}
    
    \begin{itemize}
        \item Definição dos cenários e métricas para avaliação: serão definidos
        sob quais aspectos e em quais situações o FDNS-SD deverá ser avaliado.
        \item Implementação do protocolo no Orbit Lab: O FDNS-SD será implementado
        no \textit{testbed} do Orbit Lab, descrito no capítulo \ref{ambiente}, e
        executado para coleta de dados.
        \item Análise dos resultados: depois de coletados os dados, serão gerados
        comparativos em relação aos trabalhos relacionados citados nesse artigo,
        de acordo com as métricas estabelecidas.
        \item Escrita da dissertação: a escrita da dissertação ocorrerá em paralelo
        à implementação do protocolo e coleta de dados, uma vez que novas informações
        serão adicionadas nessas fases.
        \item Apresentação da dissertação: apresentação dos resultados obtidos no
        Orbit Lab, em comparação com os resultados obtidos nos trabalhos relacionados.
    \end{itemize}

\section{Considerações finais}
\label{conclusao}

    A tradução de nomes está entre as tecnologias que menos avançaram em relação
    às MANETs. Os protocolos existentes na literatura não se mostram completamente
    adaptados às demandas diferenciadas da rede e portanto sua eficiência é limitada.
    O FDNS-SD apresenta uma nova abordagem ao problema, mas ainda necessita de
    testes para medir o seu impacto em uma rede Ad Hoc móvel.
    
    O ambiente de testes escolhido é o mais próximo possível da realidade. Falta,
    portanto, definir também cenários e métricas relevantes, que se aproximem de
    situações médias e situações extremas encaradas por MANETs, para gerar dados
    confiáveis sobre o impacto dessa nova abordagem na rede.