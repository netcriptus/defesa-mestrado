%!TEX root = ../main.tex

Este capítulo descreve a metodologia a ser aplicada para avaliação do protocolo
FDNS-SD. A seção \ref{ambiente} descreve o ambiente que será usado nos testes.

\section{Ambiente de desenvolvimento}
\label{ambiente}

    Apesar do software de testes mais utilizado para simulações de redes ser o
    \textit{Network Simulator} (NS) versão 2, a opção para testar esse protocolo
    foi outra. Para testar o \textit{FDNS-SD} em condições mais próximas às reais,
    será usado o \textit{Testbed} Orbit Lab \cite{orbit}.
    
    Operado e desenvolvido pelo laboratório \textit{WINLAB} \cite{winlab} da
    \textit{Rutgers University} em conjunto com \textit{Princeton} e
    \textit{Columbia University}, com apoio financeiro da \textit{National
    Science Foundation} (NSF), o Orbit Lab é um emulador de redes móveis, que
    conta com um grid bidimensional de 400 nós \textit{wireless} (802.11).

    A rede do Orbit Lab está preparada para simular inclusive a movimentação dos nós,
    e pode emular diversas topologias. A vantagem de testar o protocolo nesse grid é
    que os dados coletados em questões como overhead e perda de pacotes são reais, e
    não simulados, e assim é possível verificar com mais confiabilidade o comportamento
    do protocolo. O \textit{testbed} Orbit aceita apenas scripts escritos na linguagem
    Ruby.