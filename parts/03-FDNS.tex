%!TEX root = ../main.tex

Com base nos protocolos usados atualmente nas MANETs, este trabalho propõe um protocolo
completamente distribuído, que trata a tradução de nomes como um serviço na rede
em que está inserido. Esse protocolo, \textit{FDNS-SD} -- 
\textit{Fully Distributed Name Service - Service Discovery} --, atua tanto na
tradução de nomes quanto na descoberta de serviço.

Este capítulo descreve o funcionamento do \textit{FDNS-SD}, bem como os conceitos
e a terminologia usada. O protocolo sendo descrito ainda não teve seu desempenho
avaliado.

\section{Especificações}
	Este capítulo é dedicado às definições de conceitos e nomenclaturas usados pelo
	\textit{FDNS-SD}, e suas regras de funcionamento.
	
    \subsection{Conceitos iniciais}
        \begin{enumerate}
            \item \label{vizinhanca} Vizinhança -- quase todas as inundações realizadas pelo FDNS-SD são
            controladas. Inicialmente é definido um número máximo de saltos; as
            máquinas  encontradas dentro desse limite constituem a vi\-zi\-nhan\-ça
            de um nó.
            \item Servidores de nomes -- são máquinas que oferecem o serviço de
            tradução de nomes. A tradução é um serviço como qualquer outro na rede.
            Os servidores de nomes do FDNS-SD guardam o mínimo de informação
            necessária, de modo que essa informação possa ser recriada se necessário.
            Qualquer máquina pode se tornar e deixar de ser um servidor de nomes
            a qualquer momento.
            \item Time to Live (TTL) -- todas as informações são válidas por um 
            tempo máximo \textbf{T}, que depende da mobilidade e volatilidade da
            rede. Quanto mais rápida a rede, menor o tempo \textbf{T}.
            \item Tabela de traduções -- uma tabela \textit{hash} que guarda o
            endereço de cada máquina associado a um nome (que não precisa ser único)
            e uma lista de serviços oferecidos.
        \end{enumerate}
        
    \subsection{Algoritmo de criação de servidores}
        Para criar um servidor de nomes na rede, quer já exista algum ou não, os
        passos a seguir são executados.
        \begin{enumerate}
            \item Quando uma máquina \textbf{A} precisa traduzir um nome, ela
            busca em sua vizinhança se existe algum servidor de nomes. Se existir,
            ela usa esse servidor.
            \item Caso não exista servidor, essa máquina \textbf{A} torna-se um 
            servidor de nomes.
            \item \textbf{A} então faz um broadcast em sua vizinhança para montar
            sua tabela de nomes. Os vizinhos de \textbf{A} sabem que existe um
            servidor de nomes por um tempo \textbf{T}. Se o nome buscado por
            \textbf{A} for encontrado, o algoritmo chega ao fim.
            \item Caso contrário, a máquina \textbf{A} envia um broadcast à rede
            buscando outros servidores de nomes e inclui nessa mensagem qual nome
            ela quer traduzir. Qualquer servidor de nomes que receba esse broadcast
            deve guardar o IP e o nome da máquina \textbf{A} em sua tabela de nomes,
            associando a \textbf{A} um serviço de tradução de nomes e deve
            responder a ela em unicast com seu próprio nome e IP. Caso algum
            servidor conheça a tradução do nome buscado por \textbf{A}, deve
            incluir a tradução no unicast de resposta. quando \textbf{A}
            receber essa requisição, também deve responder em unicast.
        \end{enumerate}
    
    \subsection{Algoritmo de tradução de nomes}
        O seguinte algoritmo descreve como é o processo de tradução de nomes.
        \begin{enumerate}
            \item Dado que uma máquina \textbf{M} conhece um servidor de nomes
            \textbf{A} próximo a ela e \textbf{M} quer a tradução de um nome,
            então \textbf{M} envia à \textbf{A} um pedido de tradução.
            \item Caso \textbf{A} conheça a tradução pedida, devolve essa resposta
            à \textbf{M}. Caso contrário, \textbf{A} envia um multicast à todos
            os servidores em sua tabela de servidores (ou um unicast para cada
            servidor). Se algum outro servidor
            puder traduzir o nome, devolve essa resposta à \textbf{A}, que devolve
            à \textbf{M}. Caso alguma máquina tenha deixado de ser servidor mas
            ainda está na rede, informa isso à \textbf{A}. Todo servidor que não
            for capaz de traduzir o nome, simplesmente ignora a mensagem de \textbf{S}.
            \item Se receber uma resposta do multicast, \textbf{A} a repassa à
            \textbf{M} e o algoritmo acaba.
            \item Se não receber resposta, \textbf{A} inicia um broadcast buscando
            pela tradução. 
            \item Se existir, ao ser atingida pelo broadcast, a máquina buscada
            responde em unicast à \textbf{A} e
            pode se tornar um servidor de nomes. Como o nome não é único, pode
            haver mais de uma resposta. \textbf{A} deverá retornar todos os
            resultados obtivos à \textbf{M} e então o algoritmo acaba.
            \item Se não obtiver resposta, \textbf{A} retorna à \textbf{M} uma
            lista vazia, e o algoritmo acaba.
        \end{enumerate}

\section{Conflitos de nomes}
    Diferente de outras implementações do DNS, os nomes não precisam ser únicos
    no FDNS-SD. Uma tradução pode ser mais do que converter um nome em um endereço.
    Devido a essa característica, não existe conflito de nomes e uma tradução pode
    ser feita usando um dos seguintes critérios:
    
    \begin{itemize}
        \item Nome
        \item Serviço oferecido
        \item Lista de serviços oferecidos
        \item Nome e serviço oferecido
        \item Nome e lista de serviços oferecidos
        \item Endereço
    \end{itemize}
    
    O resultado de qualquer tradução é sempre uma lista contendo os endereços
    cujas características satisfaçam o critério de busca. Exceto a busca por endereço
    (conhecida no DNS como tradução reversa \cite{rfc1035}), cujo resultado deve
    ser o nome do nó referenciado por aquele endereço e uma lista dos serviços
    oferecidos. Caso o endereço buscado reversamente não exista, uma mensagem de
    erro é retornada.
    
\section{Regras de desconexão}
    Toda informação que um servidor de nomes carrega pode ser facilmente recriada
    a qualquer momento se necessário. Por esse motivo, o FDNS-SD não prevê nenhuma
    regra para que um nó se desconecte da rede, não importanto a informação que
    carrega.
    
    A ausência de um nó não precisa ser anunciada na rede. Assim como no
    \textit{Bonjour} \cite{mdns}, explicado no capítulo \ref{Bonjour}, a rede
    deve perceber de forma reativa a quebra de um enlace e decorrente indisponibilidade
    dos serviços oferecidos por aquele nó.