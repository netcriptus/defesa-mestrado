\documentclass[12pt,a4paper]{style/ufpr}

% \usepackage[portuges,brazil]{babel}
% \usepackage[portuguese,brazil]{babel}

\usepackage[brazil]{babel}
\usepackage[utf8]{inputenc}
\usepackage{amssymb,amsmath}
\usepackage{epsfig}
\usepackage{multirow}

\usepackage{amssymb}
\usepackage{subfigure}
\usepackage{graphicx}
\usepackage{caption2}
\usepackage{setspace}
\usepackage{style/ps-macros}
\usepackage{paralist}

\setcounter{secnumdepth}{3}    % n - numero de niveis de subsubsection numeradas
\setcounter{tocdepth}{3}       % coloca ate o nivel n no sumario


\title{Fully Distributed Name System-Service Discovery\\FDNS-SD}
\author{Fernando Cezar Bernardelli}
\advisortitle{Orientador}
\advisorname{Prof. Dr. Luiz Carlos Pessoa Albini}
\advisorplace{Departamento de Informática, UFPR}  % departamento, instituicao
\city{Curitiba}
\year{2013}

\date{}

\begin{document}
\renewcommand{\thesection}{\arabic{section}} 
\makecapaproposta

\tableofcontents

\begin{abstract}
    O protocolo do DNS foi um dos principais responsáveis pela popularização da
    web. A tradução de nomes para endereços possibilitou a utilização de estruturas
    humanamente legíveis para endereçamento. As redes MANET no entanto ainda pagam
    um alto preço para utilizar esse serviço: latência, alto consumo de energia e
    inconsistência são alguns dos problemas enfrentados devido ao fato do
    protocolo original ser inadequado a esse tipo de rede e das propostas de modificação
    do DNS para MANETs ainda deixarem muito a desejar. Tentando se adequar às
    restrições impostas pelas MANETs, este trabalho propõe uma nova forma de
    tradução de nomes que, apesar de se apropriar de algumas ideias do DNS
    original, foi pensada exclusivamente para uma rede Adhoc móvel.
    \\
    \\
    Palavras-chave: DNS, MANET, protocolo, rede Adhoc
\end{abstract}

\section{Introdução}
O avanço nas pesquisas de protocolos tem conferido mais confiabilidade e segurança
para redes MANET, permitindo a expansão de seu uso em vários níveis. Muitas das
barreiras e desafios identificados na última década já foram superadas ou apresentaram
uma melhora expressiva \cite{manet-state}.

Os problemas de roteamento foram resolvidos por protocolos relativamente recentes
como \textit{AdHoc On-Demand Distance Vector} (AODV) \cite{aodv} e
\textit{Dynamic Source Routing} (DSR) \cite{dsr}. A Qualidade de Serviço é
frequentemente abordada em vários novos protocolos, \cite{qos1} e \cite{qos2}
demonstram algumas pesquisas nessa área. Segurança e consumo de energia são
preocupações constantes na nova geração de protocolos, fazendo com que a rede
evolua nesses aspectos a cada ano.

\subsection{Problema}

A evolução dos protocolos de tradução de nomes tem sido tímida nessa área. Do
modo como foi projetado em 1987 \cite{rfc1035}, o protocolo original do DNS é
altamente dependente de infraestrutura. Sua especificação prevê uma hierarquia rígida,
sua segurança e confiabilidade parte do pressuposto de servidores estáveis e de
uma rede com pouca volatilidade.

Uma rede MANET elimina os principais elementos usados pelo DNS -- hierarquia em
árvore, IP's fixos associados a nomes, servidores com muitos recursos e pouca
quebra de links -- então é de se esperar que o protocolo em sua forma pura não
seja o suficiente para esse ambiente.

Diferente dos outros desafios descritos em \cite{manet-state}, não existe um
protocolo oficial ou aproximação consensualmente melhor para o problema da
resolução de nomes. Os principais protocolos que se propõe a executar a tarefa
de tradução assumem pelo menos um dos elementos citados anteriormente -- em
especial a presença de um servidor -- o que é prejudicial aos nodos na rede, ou
ao menos prejudicial aos nodos escolhidos para serem servidores.

\subsection{Objetivos}

Para propor um protocolo de tradução de nomes em uma rede que não oferece nada
da estrutura básica requerida pelo protocolo tradicional, é preciso uma nova
abordagem. O novo protocolo precisa assumir que
\begin{inparaenum}[(i)]
    \item não existe servidor fixo permanente;
    \item não existe hierarquia;
    \item um nó, assim como a informação que ele carrega pode sumir a qualquer
    instante;
    \item espaço de armazenamento pode ser limitado (uma vez que podem existir
    equipamentos com poucos recursos na rede);
    \item o número de mensagens trocadas deve ser minimizado e broadcasts devem
    ser evitados, visando economia de energia.
\end{inparaenum}

O protocolo proposto neste trabalho visa atender essas restrições e ser mais
eficiente que seus predecessores \cite{mcdns} \cite{dnssd} \cite{mdns} nas questões
de consumo de energia e espaço de armazenamento utilizado. A segurança do protocolo
no entanto está fora do escopo deste trabalho.

\subsection{Estrutura da proposta}

Este documento está organizado em quatro capítulos. O capítulo 2 analisa o estado
dos protocolos que existem hoje para resolução de nomes em MANETS, explicando o
funcionamento e apresentando os pontos fortes e fracos de cada protocolo.

O terceiro capítulo descreve a proposta de um novo protocolo completamente
distribuído para tradução de nomes e descoberta de serviços, incluindo os
algoritmos das tarefas executadas e as possibilidades de traduções. Posteriormente
algumas regras sobre conflitos de nomes e desconexão da rede são apresentadas.

O último capítulo apresenta a estratégia proposta para testes. Esse capítulo
explica o funcionamento da plataforma que será usada e por que foi escolhido um
emulador de rede ao invés de um simulador.

\section{DNS em MANET's}
%!TEX root = ../main.tex

Apesar do esforço dispendido em pesquisas sobre o \textit{DNS} para redes \textit{MANET}, ainda não existe concenso sobre qual o melhor modelo a ser adotado. Enquanto que algumas propostas exigem modificações drásticas no protocolo de resolução de nomes já existente -- impedindo assim que implementações atuais possam ser reutilizadas -- outras deixam muito a desejar no quesito de energia consumida, sobrecarregando um nodo da rede e usando-o como servidor.

Existem três casos principais para os quais as especificações de DNS para redes \textit{MANET} precisam encontrar uma solução. O primeiro é o caso de não existir nenhum servidor, ou existirem múltiplos servidores na rede. Cabe ao protocolo definir como o primeiro servidor deve ser criado, e como vai operar em relação aos demais, e se pode haver mais de um servidor. O segundo caso diz respeito à configuração dinâmica; diferente das redes com infraestrutura, não é possível ter um \textit{Master File} com nomes e traduções à disposição do Servidor de Nomes. Além disso, nomes e endereços associados aos nomes podem mudar muito em uma rede. O terceiro caso a ser tratado é o conflito de nomes -- que no caso de \textit{MANETs} pode ocorrer tanto no momento da escolha de nomes quanto na fusão de duas redes -- segundo as especificações do DNS \cite{rfc1035}, cada identificador dentro de uma rede deve ser único. A seguir são descritos os protocolos estudados e qual a sua solução para os problemas descritos.

\subsection{Bonjour}

  Criado pela \textit{Apple} primeiramente com o nome \textit{Rendezvous}, o \textit{Bonjour} implementa o \textit{Zero Configuration Network} (\textit{zeroconf}) \cite{zeroconf}, que é uma especificação que engloba vários serviços para redes com configuração dinâmica, entre eles, o \textit{Multicast DNS} (\textit{mDNS}) \cite{mdns}. Apesar do seu código ser aberto, sua documentação é limitada, portanto as especificações descritas aqui são conclusões tiradas a partir de testes realizados com o protocolo.
  
  \subsubsection{Multicast DNS}
  
    O protocolo \textit{Bonjour} é completamente reativo, e isso se aplica também à resolução dos nomes na rede, usando o \textit{mDNS}. Quando uma nova máquina entra na rede, ela anuncia via \textit{broadcast} seu nome, seu endereço \emph{IP} e os serviços que disponibiliza. As outras máquinas da rede atualizam suas tabelas de tradução de nomes e serviços. Esse método de associar nomes e serviços é chamado \textit{DNS Service Discovery} (\textit{DNS--SD}) \cite{dnssd}. Caso o nome escolhido pela nova máquina já exista na rede, essa máquina tem um número adicionado ao final do seu nome. Por exemplo, se um nó tenta entrar na rede com o nome \emph{foo}, mas esse nome já está em uso, uma mensagem é enviada a essa máquina avisando que seu nome está sendo alterado para \emph{foo1}.
    
    Uma máquina nova na rede tem sua tabela de tradução vazia, e quando precisar de algum nome ou serviço envia sua requisição para toda a rede, via \textit{multicast}. Toda máquina na rede que roda o protocolo \textit{Bonjour} é um servidor de nomes, e pode responder a essa requisição se tiver a tradução requisitada. A nova máquina atualiza sua tabela com os resultados recebidos.
    
    Se uma máquina adicionar algum serviço na rede, deve fazer um anúncio via \textit{multicast} para alertar as outras máquinas. No entanto, não há nenhum aviso se algum serviço for retirado, assim como não há nenhuma regra no protocolo para um nó sair da rede. Apesar de todas as máquinas enviarem anúncios periódicos dos serviços que disponibilizam, as entradas nas tabelas não possuem um \textit{time--to--live}, de modo que os serviços e nomes de máquinas que não estão mais na rede persistem indefinidamente.
    
    Essa abordagem funciona muito bem para redes pequenas, mas pode apresentar problemas de escalabilidade, não é apropriado para equipamentos com processamento e memória limitados. Por limitações nos testes, não sabemos se esse protocolo funciona quando é necessário fazer roteamento, isto é, quando a fonte e o destino das mensagens estão a mais de um salto de distância. O protocolo \textit{Bonjour} foi feito visando ambientes corporativos médios, por isso não existe a preocupação em escalabilidade.


\subsection{Modified Centralized DNS}

  O \textit{Modified Centralized DNS}, ou \textit{Manet DNS}, como é referenciado no seu artigo \cite{mcdns}, propõe implementar exatamente o que geralmente é evitado em projetos de protocolos para redes MANET: uma solução centralizada. O artigo alega que as afirmações de que um protocolo centralizado em uma MANET apenas apontam uma tendência esperada, mas que essa proposta nunca foi testada.
  
  O \textit{Manet DNS} se assemelha muito ao funcionamento atual do DNS para redes com infraestrutura, apesar de respeitar as normas da rede \textit{Zeroconf} \cite{zeroconf}, o que inclui configuração dinâmica, que no caso acontece sob demanda. Um Servidor de Nomes (NS) só é criado se uma máquina precisar do serviço de tradução de nome e ainda não existir um servidor na rede. Nesse caso, essa máquina se torna um servidor, e deve mandar anúncios periódicos à rede para deixar clara sua presença. Se alguma máquina não receber esse anúncio por um período de tempo, interpreta que não existe mais Servidor de Nomes em seu alcance, seja por quebra da rede -- pela mobilidade dos nós --, ou por que o antigo servidor foi desligado.
  
  Quando um NS é criado, a máquina que vai se tornar servidor envia um \textit{broadcast} à rede pedindo que todas as máquinas lhe enviem suas informações. As outras máquinas da rede respondem a essa requisição com seu nome e endereço, via \textit{unicast}, e já registram que existe um NS na rede. Caso exista um conflito de nomes, o NS aceita o nome da primeira máquina que respondeu à requisição, e envia uma mensagem para a segunda máquina pedindo para que escolha outro nome. Esse processo pode se repetir até um dado número de vezes (configurável) e se o conflito persistir, o NS escolhe um nome aleatório para essa máquina.
  
  No caso de haver mais de um servidor de DNS na rede, os NS existentes precisam chegar a um consenso sobre qual máquina vai continuar com o serviço de tradução de nomes. Essa decisão se dá comparando os IP's dos servidores; assim definindo qual tem maior prioridade sobre os demais. O servidor com maior prioridade recebe a tabela de tradução de nomes dos servidores com menos prioridade, e deve resolver eventuais conflitos de nomes que esa fusão possa causar.
  
  Se uma máquina \textbf{A}, após conseguir a tradução de uma máquina \textbf{B}, descobrir que \textbf{B} não está mais na rede, \textbf{A} pode enviar essa informação ao servidor que guarda a tradução do nome de \textbf{B}. Se o servidor receber essa informação repetidamente, retira a máquina \textbf{B} de sua tabela de tradução, considera que essa máquina não está mais disponível na rede, e o nome \textbf{B} está livre para ser usado.
  
  Os testes realizados pelo autores do protocolo revelam uma performance melhor em relação ao \textit{Multicast DNS} nos quesitos tempo de resposta e tráfego gerado por mensagens de controle. No entando os números sobre o consumo de energia do servidor não estão disponíveis, e o tempo de execução da simulação -- 100 segundos -- é considerado pequeno para esse tipo de teste.

\section{FDNS-SD}
%!TEX root = ../main.tex

Com base nos protocolos usados nas redes MANET atuais, propomos um protocolo
completamente distribuído, que trata a tradução de nomes como um serviço na rede
em que está inserido. Esse protocolo, \textit{FDNS-SD} -- 
\textit{Fully Distributed Name Service - Service Discovery} --, atua tanto em
tradução de nomes quanto em descoberta de serviço.

Este capítulo descreve o funcionamento do \textit{FDNS-SD}, bem como os conceitos
e a terminologia usada. O protocolo sendo descrito ainda é puramente teórico,
portanto seu desempenho não foi avaliado.

\subsection{Especificações}
    \subsubsection{Conceitos iniciais}
        \begin{enumerate}
            \item Vizinhança -- quase todos os floods realizados pelo FDNS-SD são
            controlados. Inicialmente é definido um número máximo de saltos; as
            máquinas  encontradas dentro desse limite constituem a vi\-zi\-nhan\-ça
            de um nó.
            \item Servidor de nomes -- são máquinas que oferecem o serviço de
            tradução de nomes. A tradução é um serviço como qualquer outro na rede.
            Os servidores de nomes do FDNS-SD guardam o mínimo de informação
            necessária, de modo que essa informação possa ser recriada se necessário.
            Qualquer máquina pode se tornar e deixar de ser um servidor de nomes
            a qualquer momento.
            \item Time to Live (TTL) -- todas as informações são válidas por um 
            tempo máximo \textbf{T}, que depende da mobilidade e volatilidade da
            rede. Quanto mais rápida a rede, menor o tempo \textbf{T}.
            \item Tabela de traduções -- uma tabela \textit{hash} que guarda o
            endereço de cada máquina associado a um nome (que não precisa ser único)
            e uma lista de serviços oferecidos.
        \end{enumerate}
        
    \subsubsection{Algoritmo de criação de servidores}
        \begin{enumerate}
            \item Quando uma máquina \textbf{A} precisa traduzir um nome, ela
            busca em sua vizinhança se existe algum servidor de nomes. Se existir,
            ela usa esse servidor.
            \item Caso não exista servidor, essa máquina \textbf{A} torna-se um 
            servidor de nomes.
            \item \textbf{A} então faz um broadcast em sua vizinhança para montar
            sua tabela de nomes. Os vizinhos de \textbf{A} sabem que existe um
            servidor de nomes por um tempo \textbf{T}. Se o nome buscado por
            \textbf{A} for encontrado, o algoritmo chega ao fim.
            \item Caso contrário, a máquina \textbf{A} envia um broadcast à rede
            buscando outros servidores de nomes, e inclui nessa mensagem qual nome
            ela quer traduzido (nome que originou todo o processo para a criação
            do servidor). Qualquer servidor de nomes que receber esse broadcast
            deve guardar o IP e o nome da máquina \textbf{A} em sua tabela de nomes,
            associando a \textbf{A} um serviço de tradução de nomes, e deve
            responder a ela em unicast com seu próprio nome e IP. Caso algum
            servidor conheça a tradução do nome buscado por \textbf{A}, deve
            incluir a tradução no unicast de resposta.
            \item Caso \textbf{A} descubra ser o único servidor de nomes na rede,
            ou não encontre a tradução do nome que procura, deve então enviar
            outro broadcast à rede com o nome buscado. Ao receber o pedido de
            tradução, a máquina com o nome procurado envia o seu IP em unicast
            para \textbf{A}, e \textbf{pode} então se tornar ela mesma um servidor
            de nomes (dependendo do seu nível de energia, por exemplo), visto que
            esse flood indica que não existem servidores de nomes próximos.
        \end{enumerate}
    
    \subsubsection{Algoritmo de tradução de nomes}
        \begin{enumerate}
            \item Dado que uma máquina \textbf{M} conhece um servidor de nomes
            \textbf{S} próximo a ela, e \textbf{M} quer a tradução de um nome,
            então \textbf{M} envia à \textbf{S} um pedido de tradução.
            \item Caso \textbf{S} conheça a tradução pedida, devolve essa resposta
            à \textbf{M}. Caso contrário, \textbf{S} envia um multicast à todos
            os servidores em sua tabela de servidores. Se algum outro servidor
            puder traduzir o nome, devolve essa resposta à \textbf{S}, que devolve
            à \textbf{M}. Caso alguma máquina tenha deixado de ser servidor mas
            ainda está na rede, informa isso à \textbf{S}. Todo servidor que não
            for capaz de traduzir o nome, simplesmente ignora a mensagem de \textbf{S}.
            \item Se receber uma resposta do multicast, \textbf{S} a repassa à
            \textbf{M}, e o algoritmo acaba.
            \item Se não receber resposta, \textbf{S} inicia um broadcast buscando
            pela tradução. 
            \item Se existir, ao ser atingida pelo broadcast, a máquina buscada
            responde em unicast à \textbf{S}, e, como já descrito anteriormente,
            pode se tornar um servidor de nomes. Como o nome não é único, pode
            haver mais de uma resposta. \textbf{S} deverá retornar todos os
            resultados obtivos à \textbf{M} e então o algoritmo acaba.
            \item Se não obtiver resposta, \textbf{S} retorna à \textbf{M} uma
            lista vazia, e o algoritmo acaba.
        \end{enumerate}

\subsection{Conflitos de nomes}
    Diferente de outras implementações do DNS, os nomes não precisam ser únicos
    no FDNS-SD. Uma tradução pode ser mais do que converter um nome em um endereço.
    Devido a essa característica, não existe conflito de nomes, e uma tradução pode
    ser feita usando um dos seguintes critérios:
    
    \begin{itemize}
        \item Nome
        \item Serviço oferecido
        \item Lista de serviços oferecidos
        \item Nome e serviço oferecido
        \item Nome e lista de serviços oferecidos
        \item Endereço
    \end{itemize}
    
    O resultado de qualquer tradução é sempre uma lista contendo os endereços
    cujas características satisfaçam o critério de busca, exceto a busca por endereço
    (conhecida no DNS como tradução reversa \cite{rfc1035}), cujo resultado deve
    ser o nome do nó referenciado por aquele endereço e uma lista dos serviços
    oferecidos, ou uma mensagem de erro caso o endereço buscado não exista.
    
\subsection{Regras de desconexão}
    Toda informação que um servidor de nomes carrega pode ser facilmente recriada
    a qualquer momento se necessário. Por esse motivo, o FDNS-SD não prevê nenhuma
    regra para que um nó se desconecte da rede, não importanto a informação que
    carrega.
    
    A ausência de um nó não precisa ser anunciada na rede. Assim como no
    \textit{Bonjour} \cite{mdns}, explicado no capítulo \ref{Bonjour}, a rede
    deve perceber de forma reativa a quebra de um link e decorrente indisponibilidade
    dos serviços oferecidos por aquele nó.

\section{Testes}
%!TEX root = ../main.tex

Para testar o \textit{FDNS-SD} em condições próximas às reais, será usado o
\textit{Testbed} Orbit Lab \cite{orbit}. Operado e desenvolvido pelo laboratório
\textit{WINLAB} \cite{winlab} da \textit{Rutgers University} em conjunto com
\textit{Princeton} e \textit{Columbia University}, com apoio financeiro da
\textit{National Science Foundation} (NSF), o Orbit Lab é um emulador de redes
móveis, que conta com um grid bidimensional de 400 nós \textit{wireless} (802.11).

A rede do Orbit Lab está preparada para simular inclusive a movimentação dos nós,
e pode emular diversas topologias. A vantagem de testar o protocolo nesse grid é
que os dados coletados em questões como overhead e perda de pacotes são reais, e
não simulados, e assim é possível verificar com mais confiabilidade o comportamento
do protocolo.

\newpage
\bibliographystyle{plain}
\bibliography{main}
\end{document}
